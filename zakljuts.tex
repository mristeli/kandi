\section{Заключение}



%Целью этой работы было изучать модальные значения в инфинитивных предложениях. Для этого мы рассматривали грамматику с точки зрения видов русского глагола сначала в общем уровне, и потом в контексте инфинитивного предложения. До этого были представлена классификация предложений в русском языке, на которой основывать остальную часть работы, и обсуждались типы видовых противопоставление настолько подробно, сколько нужно, чтобы достичь нашей цели. Наконец рассматривали именно указанный тип предложения с точки зрения модальных значений в связи с выбором вида глагола.
%
%Тема русского вида и модальности, однако, очень широкая. Эта работа лишь легко тронула эту тематику, и в её продолжении можно поставить себе разные дополнительные задачи. 
%
%Автор данной работы интересуется, прежде всего, той возможностью, как вывести указанные модальные значения от 

%-- verbioppi, lauseiden luokittelua, modaaliset merkitykset infinitiivilauseessa,

%-- jatkotutkimusta: miten modaaliset merkitykset voidaan johtaa aspektien vastakkainasettelusta, mistä modaaliset merkitykset syntyy




