\section{Заключение}

Целью этой работы было изучать модальные значения в инфинитивных предложениях. Для этого мы сначала рассматривали немного грамматики: классификация предложений и основы русского видаю Потом рассматривались некоторые примеры инфинитивного предложения. Однако, вне контекста этой работы остались много интересных языковых явлений связанных либо с употреблением видов глагола в специфических конструкциях, либо с разновидностями инфинитивных предложений. Ограничение изучения инфинитивных предложений в этой работы довольно искусственное, из-за чего не притрагивались некоторые типы данных предложений. Это, прежде всего, инфинитивные предложения с частицей бы, которые имеют свои модальные импликации. 

Над темой инфинитива и вида глагола возможно будет продолжать, уделяя больше внимания, например, типам видовых противопоставлении или значений. Вид русского глагола, наверняка, навсегда останется изучаемым, поскольку, не смотря на не короткую историю его исследования в русской лингвистике, о нём постоянно выходят новые пособия, особенно пособия, назначенные иностранцам.

Одна нехватка в этой работе является то, что в ней предмет изучения только рассматривается с помощью пособия и грамматики. Во-многом интересно было бы рассматривать живое употребление инфинитивных предложений в разговорной речи или в других текстах. Употребляется ли оно как описано в грамматической литературе и ощущается ли в нём какое-нибудь отступления, влияющие на будущий образ языка.

%До этого были представлена классификация предложений в русском языке, на которой основывать остальную часть работы, и обсуждались типы видовых противопоставление настолько подробно, сколько нужно, чтобы достичь нашей цели. Наконец рассматривали именно указанный тип предложения с точки зрения модальных  значений в связи с выбором вида глагола.
%
%Тема русского вида и модальности, однако, очень широкая. Эта работа лишь легко тронула эту тематику, и в её продолжении можно поставить себе разные дополнительные задачи. 
%
%Автор данной работы интересуется, прежде всего, той возможностью, как вывести указанные модальные значения от 

%-- verbioppi, lauseiden luokittelua, modaaliset merkitykset infinitiivilauseessa,

%-- jatkotutkimusta: miten modaaliset merkitykset voidaan johtaa aspektien vastakkainasettelusta, mistä modaaliset merkitykset syntyy




