\section{Классификация предложений}

В русском языке традиционно признаётся три типа синтаксических единиц --- \textit{словосочетания, простые предложения и сложные предложения}. Предложение составляет основную синтаксическую единицу, так как оно представляет собой, согласно Виноградову (1975: с. 254), ''грамматически оформленная по законам данного языка целостная единица речи, являющаяся главным средством формирования, выражения и сообщения мысли'', и в языке используются именно предложения для общения. Поэтому простое предложение в синтаксисе наиболее изучаемая вещь. Сложные предложения, в свою очередь, являются лишь объединения простых предложений, и их изучения концентрируется не на самых предложениях, а на способах их создании и на связи между частями таких предложений, и поэтому то, как их дальше анализируется, мы здесь не обсуждаем. Простые предложения подразделяются ещё в две группы: односоставные и двусоставные. (ЛИТ-10: с. 621 -- 622.) 

Состав простого предложения в основном определяется количеством главных член. В роли главного члена в этих предложениях может выступить либо самостоятельное сказуемое, подлежащее или и то и другое вместе. Предложения, в которых есть только один главный член (подлежащее или сказуемое: \textit{ср. Тишина. Смеркается.}), называются односоставными. Соответственно, предложения, в которых представлены и подлежащее и сказуемое), называются двусоставными (\textit{Мальчик бежит}). Из указанных групп в центре нашего внимания находятся односоставные предложения вообще и инфинитивное предложение в частности. (СОВ-02: с. 324 -- 326.)

Традиционно в русской лингвистике односоставные предложения, в свою очередь, подразделяются на следующие типы: \textit{определенно-личные, неопределенно-личные, безличные, инфинитивные и номинативные}. Определенно-личными называются предложениями, в которых главный член выражен глаголов в первом или втором лице единственного или множественного числа настоящего или будущего времени в изъявительном наклонении: \textit{читаю книгу, читаешь книгу, читаете книгу и др.}, а также в повелительном наклонении: \textit{читай книгу, читайте книгу}. 

-- here be yksiostaisten lauseiden määrittelyä ja jaottelua

\subsection{О распространенности инфинитива в русском языке}

Инфинитив --- форма глагола, несущая только чистое лексическое значение действия, обозначаемое данным глаголом. В русском языке глаголы в форме инфинитива встречаются в разных обстоятельствах. Чаще всего инфинитив употребляется в словосочетаниях, как подчиненный к другому слову член. В таких случаях инфинитив может появляться в словосочетаниях с разными членами речи. Он может примыкаться к разнообразным глаголам, существительным, прилагательным и т. д. Закономерности, касающиеся сочетания инфинитива с разными словами, часто имеют связи с лексическим значением слова, к которому подчиняется инфинитив, и поэтому с некоторыми глаголами встречаются только инфинитивы несовершенного вила, например, к фазовым глаголам (\textit{начать, продолжать, кончить и т. п.}) могут подчиняться только глаголы несовершенного вида. Соответственно, есть глаголы, к которым подчиняются только глаголы совершенного вида, например, к глаголам, которые обозначают одно целое действие с получившимся результатом, \textit{удаться, успеть}, могут подчиняться только глаголы совершенного вида. Выбор глагола может также влиять на значение словосочетаний, как например со словами \textit{нельзя, можно}, c которыми совершенный вид обозначает невозможность и несовершенный вид запрещение. (Рассудова 1968: с. 52 -- 54.)

Предметом изучения в этой работы, однако, является не случаи подчиненного к другому слова инфинитива, а так называемое инфинитивное предложение. Инфинитивными называются предложения, в которых в качестве сказуемого встречается независимый инфинитив (Тимофеев 1950: с. 262). В таких предложениях инфинитив один занимает место главного члена предложения. На место главного члена в изучаемых предложений может поступать любой глагол без лексико-семантических ограничений (АГ-80: с. 373). Поскольку в инфинитивных предложениях лишь один главный член, они входят, как выше говорилось, в число односоставных предложений в обсуждаемой классификации русских предложений (ЛИТ-10: с. 665). (см. также о классификаций предложений ЛИТ-10: с. 621.)

\subsection{Инфинитивное предложение}

-- Tähän infinitiivilauseesta kertovaa juttua valkosesta ohuesta vihkosesta "односоставные предложения в современном русском языке".



%Тимофеев (1950: с. 267) подразделяет инфинитивные предложения дальше на три: Собственно-инфинитивные предложения без частицы "бы", собственно-инфинитивные предложения с данной частицей, глагольно-инфинитивные предложения с формами глагола "быть". Эти группы отличаются именно модальными значениях. Согласно Тимофееву, собственно-инфинитивные предложения без частицы "бы" в основном выражают долженствование и необходимость, подобные предложения с данной частицей долженствование в более желательном тоне. А третья группа состоит из предложений, в которых в настоящем времени можно ощутить невидимую форму глагола быть, которого, как известно, в современном русском языке не употребляется. В рамках этой работы рассматриваются только предложения первого типа, предложения с независимым инфинитивом без служебных слов, влияющих на восприятие модальных значений данных предложений. 


Считается безличным, размышление о предикативности 

Инфинитивные предложения встречаются часто в 

Здесь будут звери и примеры и что ещё. Блин, откуда мне знать?	






Выбора глагола в инфинитивных предложениях не ограничивается лексико-семантическими правилами (АГ-90: с. 373).
%
%Ради описания модальных значений в инфинитивных предложениях следует представить, как определяется изучаемый тип предложения. В этой главе сначала обсуждается классификация предложений и других синтаксических единиц в русском языке. Классификация единиц представляется на уровне, на котором приходится для того, чтобы понято было, как определяется и ограничивается грамматическая конструкция инфинитивное предложение, рассматриваемое нами.
%
%
%Как уже обсуждали в введении синтаксис изучает языковые единицы с точки зрения оформления смысловые целые, используя которые говорящий на одном языке может передать информацию своему слушателю. Наука синтаксиса стремится к определению и классификации этих единиц. Традиционно синтаксические единицы в русском языке поделятся на три: словосочетание, простое предложение и сложное предложение (ЛИТ-10: с. 621.) Из этих трёх синтаксических единиц простое предложение является наиболее важным в свете нашей работы.

%
%\subsection{Типы предложения}
%
%
%Предложения в русском языке описываются по разным критериям, к примеру, по характеру членимости, по составу членов предложения, количеству компонентов (\todo{источник}. Однако предложения в первую очередь подразделяются на две группы, на простые и сложные предложения. 
%
%
%
%Основным функциональным признаком предложения является, таким образом, способность служить самостоятельным высказыванием. Предикативность!!
%
%Предложение представляет собой единицу связной речи -- наименьшую единицу, пригодную для общения и выражения мысли.
%
%\subsubsection{Простое предложение}
%
%
%
%%Простое предложение является основной единицей синтаксиса, потому что в предложении можно найти все %функции языка. Простое предложение всегда содержит  (cite: СРЯ част II: ст. 292.) 
%
%\subsubsection{Сложное предложение}
%
%\todo{Heppa}
%On lausetyyppi joka koostuu prost lauseista. LAuseiden välillä on joko alistus tai rinnastussuhde ja tätä kuvaa союз. Tutkimuksen kannalta ei mielenkiintoinen yleisesti ja se keskittyy lähinnä erityyppisten suhteiden luomiseen liittyviin tapoihin. 
%
%\subsection{Синтаксис простого предложения}
%
%\subsubsection{Односоставные и двусоставные предложения}
%
%Русская грамматика употребляет термины:
%Однокомпонентное - двухкомпонентное \\
%
%Вообщем-то предложенея делятся и в группы:
%Односоставное - двусоставное \\
%
%"личность", "составность", и т.п.
%
%
%Определение односоставного предложения  \\
%
%Один состав = "один главный член" -  сказуемое или подлежащее
%
%\subsubsection{Инфинитивное предложение}
%
%Один из типов односоставного предложения. Входит в число неспрягаемо-глагольного класса. Главный член - глагол в инфинитиве. Высказывает модальные значения: возможность (невозможност) - необходимость 
%
%Действие или состояние - "Вставать!" или "Здесь не пройти".
%
%Русская Грамматика страница 373 - инфинитивный класс и дальше
%
%Распространяющиц инфинитив
%
%главный член - глагол в инфинитиве
%
%Типы однокомпонентного предложения: \\ 
%	1. спрягаемо-глагольный класс \\
%	2. неспрягаемо-глагольный класс
%
%Русская грамматика ст. 348 - однокомпонентные предложения.
	