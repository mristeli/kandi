\section{Классификация предложений}

Ради описания модальных значений в инфинитивных предложениях следует представить, как определяется изучаемый тип предложения. В этой главе сначала обсуждается классификация предложений и других синтаксических единиц в русском языке. Классификация единиц представляется на уровне, на котором приходится для того, чтобы понято было, как определяется и ограничивается грамматическая конструкция инфинитивное предложение, рассматриваемое нами.

Как уже обсуждали в введении синтаксис изучает языковые единицы с точки зрения оформления смысловые целые, используя которые говорящий на одном языке может передать информацию своему слушателю. Наука синтаксиса стремится к определению и классификации этих единиц. Традиционно синтаксические единицы в русском языке поделятся на три: словосочетание, простое предложение и сложное предложение (ЛИТ-10: с. 621.) Из этих трёх синтаксических единиц простое предложение является наиболее важным в свете нашей работы.

\todo{Синтаксические единицы.}


\subsection{Типы предложения}


Предложения в русском языке описываются по разным критериям, к примеру, по характеру членимости, по составу членов предложения, количеству компонентов (\todo{источник}. Однако предложения в первую очередь подразделяются на две группы, на простые и сложные предложения. 


Основным функциональным признаком предложения является, таким образом, способность служить самостоятельным высказыванием. Предикативность!!

Предложение представляет собой единицу связной речи -- наименьшую единицу, пригодную для общения и выражения мысли.

\subsubsection{Простое предложение}



%Простое предложение является основной единицей синтаксиса, потому что в предложении можно найти все %функции языка. Простое предложение всегда содержит  (cite: СРЯ част II: ст. 292.) 

\subsubsection{Сложное предложение}

\todo{Heppa}
On lausetyyppi joka koostuu prost lauseista. LAuseiden välillä on joko alistus tai rinnastussuhde ja tätä kuvaa союз. Tutkimuksen kannalta ei mielenkiintoinen yleisesti ja se keskittyy lähinnä erityyppisten suhteiden luomiseen liittyviin tapoihin. 

\subsection{Синтаксис простого предложения}

\subsubsection{Односоставные и двусоставные предложения}

Русская грамматика употребляет термины:
Однокомпонентное - двухкомпонентное \\

Вообщем-то предложенея делятся и в группы:
Односоставное - двусоставное \\

"личность", "составность", и т.п.


Определение односоставного предложения  \\

Один состав = "один главный член" -  сказуемое или подлежащее

\subsubsection{Инфинитивное предложение}

Один из типов односоставного предложения. Входит в число неспрягаемо-глагольного класса. Главный член - глагол в инфинитиве. Высказывает модальные значения: возможность (невозможност) - необходимость 

Действие или состояние - "Вставать!" или "Здесь не пройти".

Русская Грамматика страница 373 - инфинитивный класс и дальше

Распространяющиц инфинитив

главный член - глагол в инфинитиве

Типы однокомпонентного предложения: \\ 
	1. спрягаемо-глагольный класс \\
	2. неспрягаемо-глагольный класс

Русская грамматика ст. 348 - однокомпонентные предложения.
