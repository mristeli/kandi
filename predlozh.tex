\section{Классификация предложений}

Ради описания модальных значений в инфинитивных предложениях следует представить определение изучаемого типа предложения. Причём в этой главе обсуждается классификация предложений и других синтаксических единиц в русском языке на уровне, на котором приходится для того, чтобы понято было, о какой грамматической конструкции речь идёт, когда говорим о инфинитивном предложении.

Как уже обсуждали в введении синтаксис изучает языковые единицы с точки зрения оформления смысловые целые, используя которые говорящие могут передать информацию слушателю. Наука синтаксиса стремится к определению и классификации подобных единиц. Традиционно синтаксические единицы в русском языке поделятся на три: словосочетание, простое предложение и сложное предложение (ЛИТ-10: с. 621.) Из этих трёх синтаксических единиц простое предложение в свете нашей работы является наиболее интересным. 



Tähän slovosotsetanie, prostoe pred, slozhnoe pred


\todo{Синтаксические единицы.}

Все предложения в русском языке можно описать по разным критериям, к примеру, по характеру членимости, по составу членов предложения. Однако в первую очередь все предложения делятся на две группы, на простые и  сложные предложения. 

подразделяются 

Простое предложение - сложное предложения \\
Главное предложение - придаточное предложение \\
\\

\subsection{Типы предложения}

Основным функциональным признаком предложения является, таким образом, способность служить самостоятельным высказыванием. Предикативность!!

Предложение представляет собой единицу связной речи -- наименьшую единицу, пригодную для общения и выражения мысли.

\subsubsection{Простое предложение}



%Простое предложение является основной единицей синтаксиса, потому что в предложении можно найти все %функции языка. Простое предложение всегда содержит  (cite: СРЯ част II: ст. 292.) 

\subsubsection{Сложное предложение}

\todo{Heppa}
On lausetyyppi joka koostuu prost lauseista. LAuseiden välillä on joko alistus tai rinnastussuhde ja tätä kuvaa союз. Tutkimuksen kannalta ei mielenkiintoinen yleisesti ja se keskittyy lähinnä erityyppisten suhteiden luomiseen liittyviin tapoihin. 

\subsection{Синтаксис простого предложения}

\subsubsection{Односоставные и двусоставные предложения}

Русская грамматика употребляет термины:
Однокомпонентное - двухкомпонентное \\

Вообщем-то предложенея делятся и в группы:
Односоставное - двусоставное \\

"личность", "составность", и т.п.


Определение односоставного предложения  \\

Один состав = "один главный член" -  сказуемое или подлежащее

\subsubsection{Инфинитивное предложение}

Один из типов односоставного предложения. Входит в число неспрягаемо-глагольного класса. Главный член - глагол в инфинитиве. Высказывает модальные значения: возможность (невозможност) - необходимость 

Действие или состояние - "Вставать!" или "Здесь не пройти".

Русская Грамматика страница 373 - инфинитивный класс и дальше

Распространяющиц инфинитив

главный член - глагол в инфинитиве

Типы однокомпонентного предложения: \\ 
	1. спрягаемо-глагольный класс \\
	2. неспрягаемо-глагольный класс

Русская грамматика ст. 348 - однокомпонентные предложения.
