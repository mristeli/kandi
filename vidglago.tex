\nonstopmode
% \batchmode jos et halua nähdä
\documentclass[12pt,a4paper,russian]{article}
%\documentclass{article}

\usepackage[utf8]{inputenc}
\usepackage[russian,finnish]{babel}
\usepackage[T1, T2A]{fontenc}
\usepackage{todonotes}

\title{Работа бакалавра}

%\usepackage{indentfirst}
\usepackage{color}
\usepackage[nottoc,numbib]{tocbibind}

%\usepackage{nanbib}
%\bibpunct{(}{)}{;}{a}{,}{,}

\usepackage{csquotes}
\usepackage[backend=biber,citestyle=authoryear,bibstyle=alphabetic,sorting=nyt]{biblatex}

\setlength{\parindent}{4ex}
\setlength{\parskip}{0ex} % Kappaleiden välin asetus

\linespread{1.3}

\addto\captionsrussian{% Replace "english" with the language you use
  \renewcommand{\contentsname}%
    {Оглавнение}%
}

\addto\captionsrussian{% Replace "english" with the language you use
  \renewcommand{\bibname}
    {Список использоваемой литературы}
}

\addto\captionsrussian{% Replace "english" with the language you use
  \renewcommand{\refname}
    {Список использоваемой литературы}
}

\setlength{\voffset}{-0.02in}
\setlength{\hoffset}{0.57in}
\setlength{\marginparsep}{0pt}

%\addbibresource{istocnik.bib}

\begin{document}

\begin{titlepage}
\noindent
\begin{center}	
  \setlength{\parindent}{0mm}
  \sloppy
  \large \textsc{Хельсинкский университет}
  \vspace{5mm}

  \huge \textbf{Вид глагола и модалность в инфинитивном предложении}
  \vspace{2mm}
  \textcolor{blue}\hrule
    \vspace{2mm}
	\selectlanguage{finnish}
    \large Verbin aspekti ja modaalisuus infinitiivilauseessa.
    \linebreak \vfill   
 
  \end{center}
  \vspace{15mm}
	\vfill
		
    \begin{flushright}
    	Matti Risteli \\
		Proseminaariesitelmä \\
		Venäjän kieli ja kirjallisuus \\
		Nykykielten laitos \\
		Helsingin yliopisto \\
		\today
	\end{flushright}
\end{titlepage}

\selectlanguage{russian}

\tableofcontents

\pagebreak

%%% Oma teksti tämän jälkeen.
%%% Tyhjä rivi kappalten väliin.
%%% LaTeX varaa mm. merkit & $ & % \ { } [ ] omiin tarkoituksiinsa.

\section{Введение}

Эта работа посвящена синтаксису русского языка и виду глагола. Цель работы -- это определить и описать употребление видов глагола в инфинитивных предложениях с точки зрения модальных значений изучаемых предложений. Причём тема этой работы распространяется с чистого синтаксического анализа языка даже до области прагматики.

Синтаксис --- часть грамматики, которая изучает соединения языковых единиц на смысловое целое. Языковая единица - это слова, словосочетания, предложение и т. п. Эти понятие содержат в себе те языковые единицы, которые непосредственно\todo{suora laine, korosta} служат для общения людей (АГ-80: с.5). Эти синтаксические единицы являются основой передачи мысли и информации. Изучения разновидности средств сообщения, их классификация и правила оформления являются изучаемой синтаксисом областью грамматики.

Вид глагола --- это грамматическая особенность славянских языков среди индоевропейской языковой группы. Вил глагола соответствует в нескольких значениях аспектуальность высказывания. Аспектуальность является одной из языковых универсалий, и поэтому основный смысл вида глагола в принципе легко понять. Однако употребление видов глагола не всегда просто носителям русского языка, тем не менее говорящим на русском языке, как на иностранном языке.

Модальность --- это, помимо аспектуальности, одна из языковых универсалий. Всякие предложения или высказывания имеют модальные значение, которые, кратко говоря, выражают отношение либо говорящего к сообщаемому, либо сообщаемого к действительности (АГ-80: с. 214). Способы передачи модальных значений разнообразны, что мы будем обсуждаться подробнее дальше в этой работе.

\subsection*{Структура работы}

В первой главе рассматривается классификация предложений русского языка. Обсуждается прежде всего разделение простых предложений на группы по числу главных членов, т.е., разделение на односоставные и двусоставные предложения. Из этих групп, инфинитивное предложение входит в число односоставных предложений, структуру которых мы также обсуждаем.

В второй главе рассматривается видовая система русского языка. Обсуждаются разные типы видового противопоставления, их семантические подробности и особенности выбора вида.  


\section{Классификация предложений}

В русском языке традиционно признаётся три типа синтаксических единиц --- \textit{словосочетания, простые предложения и сложные предложения}. Предложение составляет основную синтаксическую единицу, так как оно представляет собой, согласно Виноградову (1975: с. 254), ''грамматически оформленная по законам данного языка целостная единица речи, являющаяся главным средством формирования, выражения и сообщения мысли'', и в языке используются именно предложения для общения. Поэтому простое предложение в синтаксисе наиболее изучаемая вещь. Сложные предложения, в свою очередь, являются лишь объединения простых предложений, и их изучения концентрируется не на самых предложениях, а на способах их создании и на связи между частями таких предложений, и поэтому то, как их дальше анализируется, мы здесь не обсуждаем. Простые предложения подразделяются ещё в две группы: односоставные и двусоставные. (ЛИТ-10: с. 621 -- 622.) 

Состав простого предложения в основном определяется количеством главных член. В роли главного члена в этих предложениях может выступить либо самостоятельное сказуемое, подлежащее или и то и другое вместе. Предложения, в которых есть только один главный член (подлежащее или сказуемое: \textit{ср. Тишина. Смеркается.}), называются односоставными. Соответственно, предложения, в которых представлены и подлежащее и сказуемое), называются двусоставными (\textit{Мальчик бежит}). Из указанных групп в центре нашего внимания находятся односоставные предложения вообще и инфинитивное предложение в частности. (СОВ-02: с. 324 -- 326.)

Традиционно в русской лингвистике односоставные предложения, в свою очередь, подразделяются на следующие типы: \textit{определенно-личные, неопределенно-личные, безличные, инфинитивные и номинативные}. Определенно-личными называются предложениями, в которых главный член выражен глаголов в первом или втором лице единственного или множественного числа настоящего или будущего времени в изъявительном наклонении: \textit{читаю книгу, читаешь книгу, читаете книгу и др.}, а также в повелительном наклонении: \textit{читай книгу, читайте книгу}. 

-- here be yksiostaisten lauseiden määrittelyä ja jaottelua

\subsection{О распространенности инфинитива в русском языке}

Инфинитив --- форма глагола, несущая только чистое лексическое значение действия, обозначаемое данным глаголом. В русском языке глаголы в форме инфинитива встречаются в разных обстоятельствах. Чаще всего инфинитив употребляется в словосочетаниях, как подчиненный к другому слову член. В таких случаях инфинитив может появляться в словосочетаниях с разными членами речи. Он может примыкаться к разнообразным глаголам, существительным, прилагательным и т. д. Закономерности, касающиеся сочетания инфинитива с разными словами, часто имеют связи с лексическим значением слова, к которому подчиняется инфинитив, и поэтому с некоторыми глаголами встречаются только инфинитивы несовершенного вила, например, к фазовым глаголам (\textit{начать, продолжать, кончить и т. п.}) могут подчиняться только глаголы несовершенного вида. Соответственно, есть глаголы, к которым подчиняются только глаголы совершенного вида, например, к глаголам, которые обозначают одно целое действие с получившимся результатом, \textit{удаться, успеть}, могут подчиняться только глаголы совершенного вида. Выбор глагола может также влиять на значение словосочетаний, как например со словами \textit{нельзя, можно}, c которыми совершенный вид обозначает невозможность и несовершенный вид запрещение. (Рассудова 1968: с. 52 -- 54.)

Предметом изучения в этой работы, однако, является не случаи подчиненного к другому слова инфинитива, а так называемое инфинитивное предложение. Инфинитивными называются предложения, в которых в качестве сказуемого встречается независимый инфинитив (Тимофеев 1950: с. 262). В таких предложениях инфинитив один занимает место главного члена предложения. На место главного члена в изучаемых предложений может поступать любой глагол без лексико-семантических ограничений (АГ-80: с. 373). Поскольку в инфинитивных предложениях лишь один главный член, они входят, как выше говорилось, в число односоставных предложений в обсуждаемой классификации русских предложений (ЛИТ-10: с. 665). (см. также о классификаций предложений ЛИТ-10: с. 621.)

\subsection{Инфинитивное предложение}

-- Tähän infinitiivilauseesta kertovaa juttua valkosesta ohuesta vihkosesta "односоставные предложения в современном русском языке".



%Тимофеев (1950: с. 267) подразделяет инфинитивные предложения дальше на три: Собственно-инфинитивные предложения без частицы "бы", собственно-инфинитивные предложения с данной частицей, глагольно-инфинитивные предложения с формами глагола "быть". Эти группы отличаются именно модальными значениях. Согласно Тимофееву, собственно-инфинитивные предложения без частицы "бы" в основном выражают долженствование и необходимость, подобные предложения с данной частицей долженствование в более желательном тоне. А третья группа состоит из предложений, в которых в настоящем времени можно ощутить невидимую форму глагола быть, которого, как известно, в современном русском языке не употребляется. В рамках этой работы рассматриваются только предложения первого типа, предложения с независимым инфинитивом без служебных слов, влияющих на восприятие модальных значений данных предложений. 


Считается безличным, размышление о предикативности 

Инфинитивные предложения встречаются часто в 

Здесь будут звери и примеры и что ещё. Блин, откуда мне знать?	






Выбора глагола в инфинитивных предложениях не ограничивается лексико-семантическими правилами (АГ-90: с. 373).
%
%Ради описания модальных значений в инфинитивных предложениях следует представить, как определяется изучаемый тип предложения. В этой главе сначала обсуждается классификация предложений и других синтаксических единиц в русском языке. Классификация единиц представляется на уровне, на котором приходится для того, чтобы понято было, как определяется и ограничивается грамматическая конструкция инфинитивное предложение, рассматриваемое нами.
%
%
%Как уже обсуждали в введении синтаксис изучает языковые единицы с точки зрения оформления смысловые целые, используя которые говорящий на одном языке может передать информацию своему слушателю. Наука синтаксиса стремится к определению и классификации этих единиц. Традиционно синтаксические единицы в русском языке поделятся на три: словосочетание, простое предложение и сложное предложение (ЛИТ-10: с. 621.) Из этих трёх синтаксических единиц простое предложение является наиболее важным в свете нашей работы.

%
%\subsection{Типы предложения}
%
%
%Предложения в русском языке описываются по разным критериям, к примеру, по характеру членимости, по составу членов предложения, количеству компонентов (\todo{источник}. Однако предложения в первую очередь подразделяются на две группы, на простые и сложные предложения. 
%
%
%
%Основным функциональным признаком предложения является, таким образом, способность служить самостоятельным высказыванием. Предикативность!!
%
%Предложение представляет собой единицу связной речи -- наименьшую единицу, пригодную для общения и выражения мысли.
%
%\subsubsection{Простое предложение}
%
%
%
%%Простое предложение является основной единицей синтаксиса, потому что в предложении можно найти все %функции языка. Простое предложение всегда содержит  (cite: СРЯ част II: ст. 292.) 
%
%\subsubsection{Сложное предложение}
%
%\todo{Heppa}
%On lausetyyppi joka koostuu prost lauseista. LAuseiden välillä on joko alistus tai rinnastussuhde ja tätä kuvaa союз. Tutkimuksen kannalta ei mielenkiintoinen yleisesti ja se keskittyy lähinnä erityyppisten suhteiden luomiseen liittyviin tapoihin. 
%
%\subsection{Синтаксис простого предложения}
%
%\subsubsection{Односоставные и двусоставные предложения}
%
%Русская грамматика употребляет термины:
%Однокомпонентное - двухкомпонентное \\
%
%Вообщем-то предложенея делятся и в группы:
%Односоставное - двусоставное \\
%
%"личность", "составность", и т.п.
%
%
%Определение односоставного предложения  \\
%
%Один состав = "один главный член" -  сказуемое или подлежащее
%
%\subsubsection{Инфинитивное предложение}
%
%Один из типов односоставного предложения. Входит в число неспрягаемо-глагольного класса. Главный член - глагол в инфинитиве. Высказывает модальные значения: возможность (невозможност) - необходимость 
%
%Действие или состояние - "Вставать!" или "Здесь не пройти".
%
%Русская Грамматика страница 373 - инфинитивный класс и дальше
%
%Распространяющиц инфинитив
%
%главный член - глагол в инфинитиве
%
%Типы однокомпонентного предложения: \\ 
%	1. спрягаемо-глагольный класс \\
%	2. неспрягаемо-глагольный класс
%
%Русская грамматика ст. 348 - однокомпонентные предложения.
	
\section{Основы видовой системы русского языка}

В этой главе попытаемся объяснить систему вида в русском языке на уровне подробности, необходимом для изучения видов в контексте инфинитивных предложений.

\subsection{Виды глагола}

В русском языке у глаголов есть грамматическая категория вида. Категорию вида заключается в двух рядах форм глагола. Глаголы, обозначающими ограниченное целостное действие называются глаголами совершенного вида. Глаголы, которые не обладают признаком такой ограниченности, называются глаголами несовершенного вида. Категория вида присуща всем формам глагола. Важно для вас, видовое значение имеется также в инфинитиве глагола. (АГ-80 т. 1: с. 200, tarkista sivu)

Значение несовершенного вида заключается в отсутствии всякого признака ограниченности или целостности действия, обозначаемого глаголом. Глаголы несовершенного вида употребляются именно в выражении действия в его процессе или действия, стремящегося к достижению предела.

Видовые значения совершенного вида, в свою очередь, разнообразны. В основном все значение совершенного вида заключается в признаке ограниченности или целостности. Совершенным видом можно выражать достижения предела, стремление к которому выражается соответствующим глаголом несовершенного вида (\textit{Я долго читал книгу и наконец дочитал её}). Другими глаголами совершенного вида, однако, выражается достижение не результата стремления, а непроизвольного завершения (\textit{вырасти, ослабеть}). Предел, выражаемой совершенным видом может ограничивать действие также во времени, Фиксировать начало (\textit{запеть, заболеть}), окончание (\textit{отпеть} или обозначать ограничения временными пределами. (\textit{поговорить, полежать}). (там же: .) 

Большинство глаголов в русском языке составляют видовые пары: они противопоставлены друг другу по виду. Видовая пара --- пара глаголов совершенного и несовершенного вида, которые различаются только грамматической семантикой сида. Однако не все глаголы имеют видовой пары. Наличие видовой пары зависит от лексического значения глагола. (там же: с )

Здесь дальше о глаголах несовершенного вида, у которых нет видовой пары совершенного вида, но от которых можно приставками оформить глаголы совершенного вида с значением начала, окончания, ограничения временными пределами.

%присоединение к таким глаголам префиксов переводит их в группы глаголов сов. вида со значениями начала действия (закричать), действия, ограниченного временными пределами (покричать, пожить, полетать, поспать), окончания действия (отжить свой век) и с другими значениями


%\subsection{Видовые противопоставления}


\chapter{Модальность и вид глагола в инфинитивном предложении}

Модальность -- языковая 

\subsection{Средства выражения модальных значений}



АГ-80 \todo{Lue ja referoi modalnost tuosta sivulta 214-215 kahdessa kolmessa kappaleessa}

Модальность -- это одна из языковых универсалии. Каждое высказывание имеет кое-какие модальные значения. Модальные значения -- они представляют собой многие варианты. Модальность - это желаемость, необходимость, побуждителтьность, изъявительность, приказание. Модальность можно определить, как 

Модальные значения в русском языке выражаются рядом способов: интонационные конструкции, модальные слова, глагольные формы и т.д. Нише будет обсуждаться две стороны модальности, (не)возможность и необходимость в инфинитивных предложениях.

Средства формировании и выражения субъективно-модальных значений
%%% Построения с глагольными формами и, ei vaikuta relevantilta
%См. также с 231 - интонация как средство выражения субъективно-модальных значений

\todo{jatka tähän osa kuvailua infinitiivilauseissa, s. 373}

\subsection{Значение возможности}

Tee tähän esimerkkejä АГ-80:stä.

\subsection{Значение необходимости}

Tee tähän esimerkkejä АГ-80:stä.


\section{Заключение}

Целью этой работы было изучать модальные значения в инфинитивных предложениях. Для этого мы рассматривали грамматику с точки зрения видов русского глагола сначала в общем уровне, и потом в контексте инфинитивного предложения. До этого были представлена классификация предложений в русском языке, на которой основывать остальную часть работы, и обсуждались типы видовых противопоставление настолько подробно, сколько нужно, чтобы достичь нашей цели. Наконец рассматривали именно указанный тип предложения с точки зрения модальных значений в связи с выбором вида глагола.

Тема русского вида и модальности, однако, очень широкая. Эта работа лишь легко тронула эту тематику, и в её продолжении можно поставить себе разные дополнительные задачи. 

Автор данной работы интересуется, прежде всего, той возможностью, как вывести указанные модальные значения от 

%-- verbioppi, lauseiden luokittelua, modaaliset merkitykset infinitiivilauseessa,

%-- jatkotutkimusta: miten modaaliset merkitykset voidaan johtaa aspektien vastakkainasettelusta, mistä modaaliset merkitykset syntyy




%\printbibliography[heading=bibintoc,title={Список использоваемой литературы}]

\newcommand{\indenttext}{\hspace*{4ex}}
\newcommand{\cutline}{\\\indenttext}
\newcommand{\cutword}{-\cutline}

\section*{Список использоваемой литературы}
\setlength{\parindent}{0cm}
\begin{normalsize}
\subsection*{Исследовательская литература}
АГ-80 = \textit{Русская грамматика II. Синтаксис.} Главный ред. Н.Ю.\cutline Шведова. М.: Наука, 1980.\\
ВИНОГРАДОВ 1975 [1960]: Виноградов, В.В. Основные вопросы синтак\cutword сиса предложения. В.В Виноградов. \textit{Избранные труды: Исследования\cutline по русской грамматике.} М.: Наука.\\
ГЛОВИНСКАЯ 1982: Гловинская, М.Я. \textit{Семантические типы видовых\cutline противопоставлений русского глагола.} М.: Наука.\\
ЛИТ-10 = \textit{Современный русский литературный язык.} Под ред. В.Г.\cutline Костомарова, В.И. Максимова. М.: Юрайт, 2010.\\
КАЛИНИН 1994: Калинин, А.Ф. \textit{Односоставные предложения в совре\cutword менном русском языке.} Балашов: Изд-во БГПИ. \\
РАССУДОВА 1968: Рассудова, О.П. \textit{Употребление видов глагола в рус\cutline ском языке.} М.: Изд-во Московского университета.\\
СКОБЛИКОВА 2006: Скобликова, Е.С. \textit{Современный русский язык. Син-\indenttext таксис простого предложения. Учебное пособие.} М.: Флинта \& На\cutword ука.\\
CОВ-02 = \textit{Современный русский язык: Теория. Анализ языковых единиц.\cutline Ч. 2. Морфология. Синтаксис.} Под ред. Е.И. Дибровой. М.:\cutline Академия, 2002.\\
ТИМОФЕЕВ 1950: Тимофеев, К.А. Об основных типах инфинитивных \indenttext предложений в современном русском литературном языке. \textit{Вопросы \cutline синтаксиса современного русского языка.} Под ред. В.В. Виноградо\cutword ва. М.: Гос. учебно-педагог. изд-во, 257--301.\\
ТФГ = \textit{Теория функциональной грамматики т. 2: Темпоральность, \cutline Модальность.} Под ред. А.В. Бондаренко. Л.: Наука, 1990.\\
%ШАТУНОВСКИЙ 2009: Шатуновский, И.Б. \textit{Проблемы русского вида.} \cutline М.: Языки славянских культур.\\
OJANEN 1994: Ojanen, M. \textit{Грамматика: Venäjän kielioppi.} Juva: 1994 
\end{normalsize}

%%% Oma teksti ennen tätä

%\bibliographystyle{plainnat}
%\bibliographystyle{plain}
%\bibliography{gradu}  % viittaa tiedostoon gradu.bib

\end{document}
