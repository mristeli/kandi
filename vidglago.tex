\nonstopmode
% \batchmode jos et halua nähdä
% \documentclass[12pt,a4paper,russian]{article}
\documentclass{article}

\usepackage[utf8]{inputenc}
\usepackage[russian,finnish]{babel}
\usepackage[T1, T2A]{fontenc}

\title{Работа бакалавра}

\usepackage{indentfirst}
\usepackage{color}
\usepackage[nottoc,numbib]{tocbibind}

%\usepackage{nanbib}
%\bibpunct{(}{)}{;}{a}{,}{,}

\usepackage{csquotes}
\usepackage[backend=biber,citestyle=authoryear,bibstyle=alphabetic,sorting=nyt]{biblatex}

\setlength{\parindent}{4ex}
\setlength{\parskip}{0ex} % Kappaleiden välin asetus

\linespread{1.3}

\addto\captionsrussian{% Replace "english" with the language you use
  \renewcommand{\contentsname}%
    {Оглавнение}%
}

\addto\captionsrussian{% Replace "english" with the language you use
  \renewcommand{\bibname}
    {Список использоваемой литературы}
}

\addto\captionsrussian{% Replace "english" with the language you use
  \renewcommand{\refname}
    {Список использоваемой литературы}
}

\setlength{\voffset}{-0.02in}
\setlength{\hoffset}{0.57in}
\setlength{\marginparsep}{0pt}

\addbibresource{istocnik.bib}

\begin{document}

\begin{titlepage}
\noindent
\begin{center}	
  \setlength{\parindent}{0mm}
  \sloppy
  \large \textsc{Хельсинкский университет}
  \vspace{5mm}

  \huge \textbf{Вид глагола и модалность в инфинитивном предложении}
  \vspace{2mm}
  \textcolor{blue}\hrule
    \vspace{2mm}
	\selectlanguage{finnish}
    \large Verbin aspekti ja modaalisuus infinitiivilauseessa.
    \linebreak \vfill   
 
  \end{center}
  \vspace{15mm}
	\vfill
		
    \begin{flushright}
    	Matti Risteli \\
		Proseminaariesitelmä \\
		Venäjän kieli ja kirjallisuus \\
		Nykykielten laitos \\
		Helsingin yliopisto \\
		\today
	\end{flushright}
\end{titlepage}

\selectlanguage{russian}

\tableofcontents

\pagebreak

%%% Oma teksti tämän jälkeen.
%%% Tyhjä rivi kappalten väliin.
%%% LaTeX varaa mm. merkit & $ & % \ { } [ ] omiin tarkoituksiinsa.

\section{Введение}

Данная работа посвящена русскому глаголу и в нём именно противопоставлению видов глагола с точки зрения модальных значений в инфинитивных предложениях. Категория вила глагола является особенностью славянских языков среди индоевропейской языковой группы, что делает её не только трудной, но и интересной для изучения русского языка. В этой работе будут рассматриваться семантические особенности видов русского глагола сначала в общественном уровне и потом в контексте инфинитивных предложений, которые составляют хороший пример модального значения вида глагола. Кроме того будут также рассматриваться разные типы предложений, из которых для этой работы самым интересными являются односоставные предложения, к числу которых относятся и инфинитивные предложения.

%%% Тема русского глагола во многих отношениях непроста. Более того для иностранцев употребления глаголов %%% составляет больше всего проблем. Поэтому тема очень обсуждаемая и в русской и в иностранной научной %%%%% литературе. 

\subsection{Структура работы}


\section{Классификация русских предложений}

Простое предложение - сложное предложения \\
Главное предложение - придаточное предложение \\
\\

Русская грамматика употребляет термины:
Однокомпонентное - двухкомпонентное \\

Вообщем-то предложенея делятся и в группы:
Односоставное - двусоставное \\

"личность", "составность", и т.п.

\subsection{Односоставные предложения}

Определение односоставного предложения  \\

Один состав = "один главный член" -  сказуемое или подлежащее



Типы однокомпонентного предложения: \\ 
	1. спрягаемо-глагольный класс \\
	2. неспрягаемо-глагольный класс

Русская грамматика ст. 348 - однокомпонентные предложения.

\subsection{Инфинитивное предложение}

Один из типов односоставного предложения. Входит в число неспрягаемо-глагольного класса. Главный член - глагол в инфинитиве. Высказывает модальные значения: возможность (невозможност) - необходимость 

Действие или состояние - "Вставать!" или "Здесь не пройти".

Русская Грамматика страница 373 - инфинитивный класс и дальше

Распространяющиц инфинитив

главный член - глагол в инфинитиве



\section{Основы видовой системы русского языка}

\subsection{Видовое противопоставление}

\subsection{Модальность}
	Совершенный вид -	Несовершенный вид \\
	вохможность 	- 	необходимость \\
	вообще 			-	сейчас \\


\section{Модальность в инфинитивных предложениях}

Средства формировании и выражения субъективно-модальных значений
%%% Построения с глагольными формами и, ei vaikuta relevantilta

См. также с 231 - интонация как средство выражения субъективно-модальных значений


\subsection{Значение возможности}

\subsection{Значение необходимости}



\section{Выводы}

Выше были рассмотрены некоторые типы употребления русского вида. Внимание направлялось на модалность высказываний прежде всего в инфинитивных предложениях.

Употребления видов с точки зрения модальных значений совсем не 



\printbibliography[heading=bibintoc,title={Список использоваемой литературы}]

%%% Oma teksti ennen tätä

%\bibliographystyle{plainnat}
%\bibliographystyle{plain}
%\bibliography{gradu}  % viittaa tiedostoon gradu.bib

\end{document}
