\nonstopmode
% \batchmode jos et halua nähdä
% \documentclass[12pt,a4paper,russian]{article}
\documentclass{article}

\usepackage[utf8]{inputenc}
\usepackage[russian,finnish]{babel}
\usepackage[T1, T2A]{fontenc}

\title{Работа бакалавра}

\usepackage{indentfirst}
\usepackage{color}
\usepackage[nottoc,numbib]{tocbibind}

%\usepackage{nanbib}
%\bibpunct{(}{)}{;}{a}{,}{,}

\usepackage{csquotes}
\usepackage[backend=biber,citestyle=authoryear,bibstyle=alphabetic,sorting=nyt]{biblatex}

\setlength{\parindent}{4ex}
\setlength{\parskip}{0ex} % Kappaleiden välin asetus

\linespread{1.3}

\addto\captionsrussian{% Replace "english" with the language you use
  \renewcommand{\contentsname}%
    {Оглавнение}%
}

\addto\captionsrussian{% Replace "english" with the language you use
  \renewcommand{\bibname}
    {Список использоваемой литературы}
}

\addto\captionsrussian{% Replace "english" with the language you use
  \renewcommand{\refname}
    {Список использоваемой литературы}
}

\setlength{\voffset}{-0.02in}
\setlength{\hoffset}{0.57in}
\setlength{\marginparsep}{0pt}

\addbibresource{istocnik.bib}

\begin{document}

\begin{titlepage}
\noindent
\begin{center}	
  \setlength{\parindent}{0mm}
  \sloppy
  \large \textsc{Хельсинкский университет}
  \vspace{5mm}

  \huge \textbf{Вид глагола и модалность в инфинитивном предложении}
  \vspace{2mm}
  \textcolor{blue}\hrule
    \vspace{2mm}
	\selectlanguage{finnish}
    \large Verbin aspekti ja modaalisuus infinitiivilauseessa.
    \linebreak \vfill   
 
  \end{center}
  \vspace{15mm}
	\vfill
		
    \begin{flushright}
    	Matti Risteli \\
		Proseminaariesitelmä \\
		Venäjän kieli ja kirjallisuus \\
		Nykykielten laitos \\
		Helsingin yliopisto \\
		\today
	\end{flushright}
\end{titlepage}

\selectlanguage{russian}

\tableofcontents

\pagebreak

%%% Oma teksti tämän jälkeen.
%%% Tyhjä rivi kappalten väliin.
%%% LaTeX varaa mm. merkit & $ & % \ { } [ ] omiin tarkoituksiinsa.

\section*{Введение}

%%% Тема русского глагола во многих отношениях непроста. Более того для иностранцев употребления глаголов %%% составляет больше всего проблем. Поэтому тема очень обсуждаемая и в русской и в иностранной научной %%%%% литературе. 

\subsection*{Структура работы}

В первой главе рассматривается классификация предложений русского языка. Обсуждается прежде всего разделение простых предложений на группы по числу главных членов, т.е., разделение на односоставные и двусоставные предложения. Из этих групп, инфинитивное предложение входит в число односоставных предложений.

В второй главе рассматривается видовая система русского языка. Обсуждаются разные типы видового противопоставления, их семантические подробности и отношение выбора вида к выражению модальности.  

В третьей главе представляются инфинитивные предложения, и рассматривается их модальные значения указанных высказываний.

\section{Классификация русских предложений}

Все предложения в русском языке можно описать по разным критериям, к примеру, по характеру членимости, по составу членов предложения. Однако в первую очередь все предложения делятся на две группы, на простые и  сложные предложения. 

\subsection{Простое предложение}

Простое предложение является основной единицей синтаксиса, потому что в предложении можно найти все функции языка.  (cite: СРЯ част II: ст. 292.) 


\subsubsection{Односоставное предложения}

\subsubsection{Двусоставное предложения}

\subsection{Сложное предложение}

подразделяются 

Простое предложение - сложное предложения \\
Главное предложение - придаточное предложение \\
\\

Русская грамматика употребляет термины:
Однокомпонентное - двухкомпонентное \\

Вообщем-то предложенея делятся и в группы:
Односоставное - двусоставное \\

"личность", "составность", и т.п.

\subsection{Односоставные предложения}

Определение односоставного предложения  \\

Один состав = "один главный член" -  сказуемое или подлежащее



Типы однокомпонентного предложения: \\ 
	1. спрягаемо-глагольный класс \\
	2. неспрягаемо-глагольный класс

Русская грамматика ст. 348 - однокомпонентные предложения.

\subsection{Инфинитивное предложение}

Один из типов односоставного предложения. Входит в число неспрягаемо-глагольного класса. Главный член - глагол в инфинитиве. Высказывает модальные значения: возможность (невозможност) - необходимость 

Действие или состояние - "Вставать!" или "Здесь не пройти".

Русская Грамматика страница 373 - инфинитивный класс и дальше

Распространяющиц инфинитив

главный член - глагол в инфинитиве



\section{Основы видовой системы русского языка}

\subsection{Видовое противопоставление}

\subsection{Модальность}
	Совершенный вид -	Несовершенный вид \\
	вохможность 	- 	необходимость \\
	вообще 			-	сейчас \\


\section{Модальность в инфинитивных предложениях}

Средства формировании и выражения субъективно-модальных значений
%%% Построения с глагольными формами и, ei vaikuta relevantilta

См. также с 231 - интонация как средство выражения субъективно-модальных значений


\subsection{Значение возможности}

\subsection{Значение необходимости}



\section{Заключение}

Целью этой работы было изучать модальные значения в инфинитивных предложениях. Для этого мы рассматривали грамматику с точки зрения видов русского глагола сначала в общем уровне, и потом в контексте инфинитивного предложения. До этого были представлена классификация предложений в русском языке, на которой основывать остальную часть работы, и обсуждались типы видовых противопоставление настолько подробно, сколько нужно, чтобы достичь нашей цели. Наконец рассматривали именно указанный тип предложения с точки зрения модальных значений в связи с выбором вида глагола.

Тема русского вида и модальности, однако, очень широкая. Эта работа лишь легко тронула эту тематику, и в её продолжении можно поставить себе разные дополнительные задачи. 

Автор данной работы интересуется, прежде всего, той возможностью, как вывести указанные модальные значения от 

%-- verbioppi, lauseiden luokittelua, modaaliset merkitykset infinitiivilauseessa,

%-- jatkotutkimusta: miten modaaliset merkitykset voidaan johtaa aspektien vastakkainasettelusta, mistä modaaliset merkitykset syntyy




\printbibliography[heading=bibintoc,title={Список использоваемой литературы}]

%%% Oma teksti ennen tätä

%\bibliographystyle{plainnat}
%\bibliographystyle{plain}
%\bibliography{gradu}  % viittaa tiedostoon gradu.bib

\end{document}
