\nonstopmode
% \batchmode jos et halua nähdä
% \documentclass[12pt,a4paper,russian]{article}
\documentclass{article}

\usepackage[utf8]{inputenc}
\usepackage[russian,finnish]{babel}
\usepackage[T1, T2A]{fontenc}
\usepackage{todonotes}

\title{Работа бакалавра}

%\usepackage{indentfirst}
\usepackage{color}
\usepackage[nottoc,numbib]{tocbibind}

%\usepackage{nanbib}
%\bibpunct{(}{)}{;}{a}{,}{,}

\usepackage{csquotes}
\usepackage[backend=biber,citestyle=authoryear,bibstyle=alphabetic,sorting=nyt]{biblatex}

\setlength{\parindent}{4ex}
\setlength{\parskip}{0ex} % Kappaleiden välin asetus

\linespread{1.3}

\addto\captionsrussian{% Replace "english" with the language you use
  \renewcommand{\contentsname}%
    {Оглавнение}%
}

\addto\captionsrussian{% Replace "english" with the language you use
  \renewcommand{\bibname}
    {Список использоваемой литературы}
}

\addto\captionsrussian{% Replace "english" with the language you use
  \renewcommand{\refname}
    {Список использоваемой литературы}
}

\setlength{\voffset}{-0.02in}
\setlength{\hoffset}{0.57in}
\setlength{\marginparsep}{0pt}

\addbibresource{istocnik.bib}

\begin{document}

\begin{titlepage}
\noindent
\begin{center}	
  \setlength{\parindent}{0mm}
  \sloppy
  \large \textsc{Хельсинкский университет}
  \vspace{5mm}

  \huge \textbf{Вид глагола и модалность в инфинитивном предложении}
  \vspace{2mm}
  \textcolor{blue}\hrule
    \vspace{2mm}
	\selectlanguage{finnish}
    \large Verbin aspekti ja modaalisuus infinitiivilauseessa.
    \linebreak \vfill   
 
  \end{center}
  \vspace{15mm}
	\vfill
		
    \begin{flushright}
    	Matti Risteli \\
		Proseminaariesitelmä \\
		Venäjän kieli ja kirjallisuus \\
		Nykykielten laitos \\
		Helsingin yliopisto \\
		\today
	\end{flushright}
\end{titlepage}

\selectlanguage{russian}

\tableofcontents

\pagebreak

%%% Oma teksti tämän jälkeen.
%%% Tyhjä rivi kappalten väliin.
%%% LaTeX varaa mm. merkit & $ & % \ { } [ ] omiin tarkoituksiinsa.

\section{Введение}

%%% Тема русского глагола во многих отношениях непроста. Более того для иностранцев употребления глаголов %%% составляет больше всего проблем. Поэтому тема очень обсуждаемая и в русской и в иностранной научной %%%%% литературе. 

Эта работа посвящена грамматике русского языка. Цель работы -- это определить и описать употребление видов глагола в так называемых инфинитивных предложениях с точки зрения модальных значений изучаемых предложений. Причём тема этой работы распространяется с чистого синтаксического анализа языка до прагматики.

Синтаксис --- часть грамматики, которая изучает соединения языковых единиц на смысловое целое. Языковая единица - это слова, словосочетания, предложение и т. п. Эти понятие содержат в себе те языковые единицы, которые непосредственно\todo{suora laine, korosta} служат для общения людей (АГ-80: с.5). Эти синтаксические единицы являются основой передачи мысли и информации. Изучения разновидности средств сообщения, их классификация и правила оформления являются изучаемой синтаксисом областью грамматики.

Модальность - это одна из языковых универсалий. Все высказывания имеют кое-какие модальные значения. Модальными значениями имеются в виду значения, как например "" 


\subsection*{Структура работы}

В первой главе рассматривается классификация предложений русского языка. Обсуждается прежде всего разделение простых предложений на группы по числу главных членов, т.е., разделение на односоставные и двусоставные предложения. Из этих групп, инфинитивное предложение входит в число односоставных предложений.

В второй главе рассматривается видовая система рус ского языка. Обсуждаются разные типы видового противопоставления, их семантические подробности и отношение выбора вида к выражению модальности.  

В третьей главе представляются инфинитивные предложения, и рассматривается их модальные значения указанных высказываний.


\section{Классификация предложений}


 Синтаксические единицы в русском языке традиционно поделятся на три: словосочетание, простое предложение и сложное предложение (ЛИТ-10: с. 621.) Из этих трёх синтаксических единиц простое предложение в свете нашей работы является наиболее интересным. 



Tähän slovosotsetanie, prostoe pred, slozhnoe pred


\todo{Синтаксические единицы.}

Все предложения в русском языке можно описать по разным критериям, к примеру, по характеру членимости, по составу членов предложения. Однако в первую очередь все предложения делятся на две группы, на простые и  сложные предложения. 

подразделяются 

Простое предложение - сложное предложения \\
Главное предложение - придаточное предложение \\
\\

\subsection{Типы предложения}

Основным функциональным признаком предложения является, таким образом, способность служить самостоятельным высказыванием. Предикативность!!

Предложение представляет собой единицу связной речи -- наименьшую единицу, пригодную для общения и выражения мысли.

\subsubsection{Простое предложение}



%Простое предложение является основной единицей синтаксиса, потому что в предложении можно найти все %функции языка. Простое предложение всегда содержит  (cite: СРЯ част II: ст. 292.) 

\subsubsection{Сложное предложение}

\todo{Heppa}
On lausetyyppi joka koostuu prost lauseista. LAuseiden välillä on joko alistus tai rinnastussuhde ja tätä kuvaa союз. Tutkimuksen kannalta ei mielenkiintoinen yleisesti ja se keskittyy lähinnä erityyppisten suhteiden luomiseen liittyviin tapoihin. 

\subsection{Синтаксис простого предложения}

\subsubsection{Односоставные и двусоставные предложения}

Русская грамматика употребляет термины:
Однокомпонентное - двухкомпонентное \\

Вообщем-то предложенея делятся и в группы:
Односоставное - двусоставное \\

"личность", "составность", и т.п.


Определение односоставного предложения  \\

Один состав = "один главный член" -  сказуемое или подлежащее

\subsubsection{Инфинитивное предложение}

Один из типов односоставного предложения. Входит в число неспрягаемо-глагольного класса. Главный член - глагол в инфинитиве. Высказывает модальные значения: возможность (невозможност) - необходимость 

Действие или состояние - "Вставать!" или "Здесь не пройти".

Русская Грамматика страница 373 - инфинитивный класс и дальше

Распространяющиц инфинитив

главный член - глагол в инфинитиве

Типы однокомпонентного предложения: \\ 
	1. спрягаемо-глагольный класс \\
	2. неспрягаемо-глагольный класс

Русская грамматика ст. 348 - однокомпонентные предложения.

%\section{Основы видовой системы русского языка}
\section{Основы видов глагола в русском языке}

\subsection{Вид глагола}

\subsection{Видовые противопоставления}

\section{Модальность в инфинитивном предложении}

\subsection{Средства выражения модальных значений}

АГ-80 \todo{Lue ja referoi modalnost tuosta sivulta 214-215 kahdessa kolmessa kappaleessa}

Модальность -- это одна из языковых универсалии. Каждое высказывание имеет кое-какие модальные значения. Модальные значения -- они представляют собой многие варианты. Модальность - это желаемость, необходимость, побуждителтьность, изъявительность, приказание. Модальность можно определить, как 

Модальные значения в русском языке выражаются рядом способов: интонационные конструкции, модальные слова, глагольные формы и т.д. Нише будет обсуждаться две стороны модальности, (не)возможность и необходимость в инфинитивных предложениях.

Средства формировании и выражения субъективно-модальных значений
%%% Построения с глагольными формами и, ei vaikuta relevantilta
%См. также с 231 - интонация как средство выражения субъективно-модальных значений

\todo{jatka tähän osa kuvailua infinitiivilauseissa, s. 373}

\subsection{Значение возможности}

Tee tähän esimerkkejä АГ-80:stä.

\subsection{Значение необходимости}

Tee tähän esimerkkejä АГ-80:stä.

\section{Заключение}

Целью этой работы было изучать модальные значения в инфинитивных предложениях. Для этого мы рассматривали грамматику с точки зрения видов русского глагола сначала в общем уровне, и потом в контексте инфинитивного предложения. До этого были представлена классификация предложений в русском языке, на которой основывать остальную часть работы, и обсуждались типы видовых противопоставление настолько подробно, сколько нужно, чтобы достичь нашей цели. Наконец рассматривали именно указанный тип предложения с точки зрения модальных значений в связи с выбором вида глагола.

Тема русского вида и модальности, однако, очень широкая. Эта работа лишь легко тронула эту тематику, и в её продолжении можно поставить себе разные дополнительные задачи. 

Автор данной работы интересуется, прежде всего, той возможностью, как вывести указанные модальные значения от 

%-- verbioppi, lauseiden luokittelua, modaaliset merkitykset infinitiivilauseessa,

%-- jatkotutkimusta: miten modaaliset merkitykset voidaan johtaa aspektien vastakkainasettelusta, mistä modaaliset merkitykset syntyy




%\printbibliography[heading=bibintoc,title={Список использоваемой литературы}]

\newcommand{\indenttext}{\hspace*{4ex}}
\newcommand{\cutline}{\\\indenttext}
\newcommand{\cutword}{-\cutline}

\section*{Список использованной литературы}
\setlength{\parindent}{0cm}
\begin{normalsize}
\subsection*{Исследовательская литература}
АГ-80 = \textit{Русская грамматика I \& II.} Главный ред. Н.Ю.\cutline Шведова. М.: Наука, 1980.\\
ВИНОГРАДОВ 1975 [1960]: Виноградов, В.В. Основные вопросы синтак\cutword сиса предложения. В.В Виноградов. \textit{Избранные труды: Исследования\cutline по русской грамматике.} М.: Наука.\\
ГЛОВИНСКАЯ 1982: Гловинская, М.Я. \textit{Семантические типы видовых\cutline противопоставлений русского глагола.} М.: Наука.\\
ЛИТ-10 = \textit{Современный русский литературный язык.} Под ред. В.Г.\cutline Костомарова, В.И. Максимова. М.: Юрайт, 2010.\\
КАЛИНИН 1994: Калинин, А.Ф. \textit{Односоставные предложения в совре\cutword менном русском языке.} Балашов: Изд-во БГПИ. \\
РАССУДОВА 1968: Рассудова, О.П. \textit{Употребление видов глагола в рус\cutline ском языке.} М.: Изд-во Московского университета.\\
СКОБЛИКОВА 2006: Скобликова, Е.С. \textit{Современный русский язык. Син-\indenttext таксис простого предложения. Учебное пособие.} М.: Флинта \& На\cutword ука.\\
CОВ-02 = \textit{Современный русский язык: Теория. Анализ языковых единиц.\cutline Ч. 2. Морфология. Синтаксис.} Под ред. Е.И. Дибровой. М.:\cutline Академия, 2002.\\
ТИМОФЕЕВ 1950: Тимофеев, К.А. Об основных типах инфинитивных \indenttext предложений в современном русском литературном языке. \textit{Вопросы \cutline синтаксиса современного русского языка.} Под ред. В.В. Виноградо\cutword ва. М.: Гос. учебно-педагог. изд-во, 257--301.\\
ТФГ = \textit{Теория функциональной грамматики т. 2: Темпоральность, \cutline Модальность.} Под ред. А.В. Бондаренко. Л.: Наука, 1990.\\
%ШАТУНОВСКИЙ 2009: Шатуновский, И.Б. \textit{Проблемы русского вида.} \cutline М.: Языки славянских культур.\\
\\
%OJANEN 1994: Ojanen, M. \textit{Грамматика: Venäjän kielioppi.} Juva: 1994 
\end{normalsize}

%%% Oma teksti ennen tätä

%\bibliographystyle{plainnat}
%\bibliographystyle{plain}
%\bibliography{gradu}  % viittaa tiedostoon gradu.bib

\end{document}
