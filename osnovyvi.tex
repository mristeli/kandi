\section{Основы видовой системы русского языка}

В этой главе попытаемся объяснить систему вида в русском языке на уровне подробности, необходимом для изучения видов в контексте инфинитивных предложений.

\subsection{Виды глагола}

В русском языке у глаголов есть грамматическая категория вида. Категорию вида заключается в двух рядах форм глагола. Глаголы, обозначающими ограниченное целостное действие называются глаголами совершенного вида. Глаголы, которые не обладают признаком такой ограниченности, называются глаголами несовершенного вида. Категория вида присуща всем формам глагола. Важно для вас, видовое значение имеется также в инфинитиве глагола. (АГ-80 т. 1: с. 200, tarkista sivu)

Значение несовершенного вида заключается в отсутствии всякого признака ограниченности или целостности действия, обозначаемого глаголом. Глаголы несовершенного вида употребляются именно в выражении действия в его процессе или действия, стремящегося к достижению предела.

Видовые значения совершенного вида, в свою очередь, разнообразны. В основном все значение совершенного вида заключается в признаке ограниченности или целостности. Совершенным видом можно выражать достижения предела, стремление к которому выражается соответствующим глаголом несовершенного вида (\textit{Я долго читал книгу и наконец дочитал её}). Другими глаголами совершенного вида, однако, выражается достижение не результата стремления, а непроизвольного завершения (\textit{вырасти, ослабеть}). Предел, выражаемой совершенным видом может ограничивать действие также во времени, Фиксировать начало (\textit{запеть, заболеть}), окончание (\textit{отпеть} или обозначать ограничения временными пределами. (\textit{поговорить, полежать}). (там же: .) 

Большинство глаголов в русском языке составляют видовые пары: они противопоставлены друг другу по виду. Видовая пара --- пара глаголов совершенного и несовершенного вида, которые различаются только грамматической семантикой сида. Однако не все глаголы имеют видовой пары. Наличие видовой пары зависит от лексического значения глагола. (там же: с )

Здесь дальше о глаголах несовершенного вида, у которых нет видовой пары совершенного вида, но от которых можно приставками оформить глаголы совершенного вида с значением начала, окончания, ограничения временными пределами.

%присоединение к таким глаголам префиксов переводит их в группы глаголов сов. вида со значениями начала действия (закричать), действия, ограниченного временными пределами (покричать, пожить, полетать, поспать), окончания действия (отжить свой век) и с другими значениями


%\subsection{Видовые противопоставления}

