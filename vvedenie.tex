\section{Введение}
Эта работа посвящена грамматике русского языка. В работе рассматриваются инфинитивные предложения с целью определить и описать употребление видов глагола в данного типа предложения с точки зрения некоторых модальных значений. Изучение предложений относится к области грамматика, называющейся синтаксис. Это раздел грамматики, изучающий соединения языковых единиц на смысловые целые. В языковые единицы входят, к примеру, слова, словосочетания, предложения и т. п. Эти понятия содержат в основные средства, которые \textit{непосредственно} служат для общения людей (АГ-80: с. 5). Любая передача мысли и информации основывается на синтаксических единицах (СОВ-10: с. \todo{check out}). Разновидности средств сообщения, их классификация и закономерности оформления разных конструкции --- именно предмет изучения синтаксиса. Поскольку предмет изучения в этой работы --- значения инфинитивных предложений, наша работа, вернее всего, относится именно к синтаксису.
%
%Вид глагола --- это грамматическая особенность славянских языков. В них, в том числе и в русском языке, большинство глаголов имеет два варианта, вариант совершенного вида (\textit{СВ}) и несовершенного вида (\textit{НСВ}). Вид глагола во многих отношениях соответствует грамматическому аспекту высказывания. Аспектуальность является одной из языковых универсалий. Однако употребление видов глагола не всегда просто даже носителям русского языка, тем более говорящим на русском языке, как на иностранном языке.
%
%Модальность --- это, помимо аспектуальности, одна из языковых универсалий. Всякие предложения или высказывания на любом языке имеют модальные значение, которые, кратко говоря, выражают отношение либо говорящего к сообщаемому, либо сообщаемого к действительности (АГ-80: с. 214). Способы передачи модальных значений разнообразны, грамматические, интонационные и т. п. мы будем обсуждать лишь некоторые из них подробнее нише в этой работе.

\subsection*{Структура работы}

В первой главе рассматривается предмет изучения этой работы --- инфинитивное предложение. Обсуждается определение данного типа предложения и рассматривается также некоторые близки к нему грамматические понятия и закономерности. Во второй главе легко рассматривается видовая система русского языка. Изучаются типы видового противопоставления и значении, несомые СВ и НСВ. В третьей главе описываются некоторые типы инфинитивных предложений.  Анализ этих предложений ведётся, как указано выше, с точки зрения их модальных значений в связи с выбором вида глагола. Показываются примеры значений необходимости и (не)возможности и сравниваем их с указанными нами во второй главе закономерности и значения видов глагола.