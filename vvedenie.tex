\section{Введение}

Эта работа посвящена виду глагола и синтаксису русского языка. Цель работы -- это определить и описать употребление видов глагола в инфинитивных предложениях с точки зрения модальных значений изучаемых предложений. Причём тема этой работы распространяется с чистого синтаксического анализа языка до области прагматики.

Синтаксис --- часть грамматики, которая изучает соединения языковых единиц на смысловые целые. Языковая единица --- это слова, словосочетания, предложение и т. п. Эти понятия содержат в себе те языковые единицы, которые \textit{непосредственно} служат для общения людей (АГ-80: с. 5). Это обозначает, что синтаксические единицы являются основой передачи мысли и информации, и поэтому основные единицы (СОВ-10: с. \todo{check out}). Изучение разновидности средств сообщения, их классификация и правила оформления именно является изучаемой синтаксисом областью грамматики.

Вид глагола --- это грамматическая особенность славянских языков среди индоевропейской языковой группы. В славянских языках, в том числе и в русском языке, в-основном почти все глаголы имеют два варианта, вариант совершенного вида (\textit{СВ}) и несовершенного вида (\textit{НСВ}). Вил глагола во многих отношениях соответствует грамматический аспект высказывания. Аспектуальность является одной из языковых универсалий, и поэтому основные правила употребления видов глагола в принципе легко понять (\todo{источник?}). Однако употребление видов глагола не всегда просто даже носителям русского языка, тем более говорящим на русском языке, как на иностранном языке, и поэтому тема русского вида очень популярна среди языковедов-русистов.

Модальность --- это, помимо аспектуальности, также одна из языковых универсалий. Всякие предложения или высказывания на любом языке имеют модальные значение, которые, кратко говоря, выражают отношение либо говорящего к сообщаемому, либо сообщаемого к действительности (АГ-80: с. 214). Способы передачи модальных значений разнообразны, и мы будем обсуждать их подробнее нише в этой работе.

\subsection*{Структура работы}

В первой главе рассматривается классификация предложений русского языка. Обсуждается прежде всего разделение простых предложений на группы по числу главных членов, т.е., разделение на односоставные и двусоставные предложения. Из этих групп, инфинитивное предложение входит в число односоставных предложений, структуру которых мы также обсуждаем.

В второй главе рассматривается видовая система русского языка.	 Обсуждаются разные типы видового противопоставления, их семантические подробности и особенности выбора вида.  

