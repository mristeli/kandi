\section{Введение}

Эта работа посвящена синтаксису русского языка и виду глагола. Цель работы -- это определить и описать употребление видов глагола в инфинитивных предложениях с точки зрения модальных значений изучаемых предложений. Причём тема этой работы распространяется с чистого синтаксического анализа языка даже до области прагматики.

Синтаксис --- часть грамматики, которая изучает соединения языковых единиц на смысловое целое. Языковая единица - это слова, словосочетания, предложение и т. п. Эти понятие содержат в себе те языковые единицы, которые непосредственно\todo{suora laine, korosta} служат для общения людей (АГ-80: с.5). Эти синтаксические единицы являются основой передачи мысли и информации. Изучения разновидности средств сообщения, их классификация и правила оформления являются изучаемой синтаксисом областью грамматики.

Вид глагола --- это грамматическая особенность славянских языков среди индоевропейской языковой группы. Вил глагола соответствует в нескольких значениях аспектуальность высказывания. Аспектуальность является одной из языковых универсалий, и поэтому основный смысл вида глагола в принципе легко понять. Однако употребление видов глагола не всегда просто носителям русского языка, тем не менее говорящим на русском языке, как на иностранном языке.

Модальность --- это, помимо аспектуальности, одна из языковых универсалий. Всякие предложения или высказывания имеют модальные значение, которые, кратко говоря, выражают отношение либо говорящего к сообщаемому, либо сообщаемого к действительности (АГ-80: с. 214). Способы передачи модальных значений разнообразны, что мы будем обсуждаться подробнее дальше в этой работе.

\subsection*{Структура работы}

В первой главе рассматривается классификация предложений русского языка. Обсуждается прежде всего разделение простых предложений на группы по числу главных членов, т.е., разделение на односоставные и двусоставные предложения. Из этих групп, инфинитивное предложение входит в число односоставных предложений, структуру которых мы также обсуждаем.

В второй главе рассматривается видовая система русского языка. Обсуждаются разные типы видового противопоставления, их семантические подробности и особенности выбора вида.  

