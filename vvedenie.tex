\section{Введение}

Эта работа чисто посвящена грамматике русского языка. В ней интересуемся именно x§классификацией предложений и употреблением видов глагола. Точно говоря, цель работы -- это определить и описать употребление видов глагола в инфинитивных предложениях нескольких типов с точки зрения модальных значений изучаемых предложений.

Синтаксис --- часть грамматики, которая изучает соединения языковых единиц на смысловые целые. Языковая единица --- это слова, словосочетания, предложение и т. п. Эти понятия содержат в себе те языковые единицы, которые \textit{непосредственно} служат для общения людей (АГ-80: с. 5). Это обозначает, что синтаксические единицы являются основой передачи мысли и информации, и поэтому основные единицы (СОВ-10: с. \todo{check out}). Изучение разновидности средств сообщения, их классификация и правила оформления именно является изучаемой синтаксисом областью грамматики.

Вид глагола --- это грамматическая особенность славянских языков среди индоевропейской языковой группы. В славянских языках, в том числе и в русском языке, почти все глаголы имеют два варианта, вариант совершенного вида (\textit{СВ}) и несовершенного вида (\textit{НСВ}). Вид глагола во многих отношениях соответствует грамматический аспект высказывания. Аспектуальность является одной из языковых универсалий, и поэтому основные правила употребления видов глагола в принципе легко понять (\todo{источник?}). Однако употребление видов глагола не всегда просто даже носителям русского языка, тем более говорящим на русском языке, как на иностранном языке.

Модальность --- это, помимо аспектуальности, одна из языковых универсалий. Всякие предложения или высказывания на любом языке имеют модальные значение, которые, кратко говоря, выражают отношение либо говорящего к сообщаемому, либо сообщаемого к действительности (АГ-80: с. 214). Способы передачи модальных значений разнообразны, и мы будем обсуждать их подробнее нише в этой работе.

\subsection{Структура работы}

В первой главе рассматривается классификация предложений русского языка. Обсуждается прежде всего разделение простых предложений на группы по числу главных членов, т.е., разделение на односоставные и двусоставные предложения, что является актуально для этой работы в определении объекта исследования. Из этих групп, изучаемое нами инфинитивное предложение входит в число односоставных предложений.

Во второй главе рассматривается видовая система русского языка. Поскольку целое описание видовой системы само по себе сложное и вне контекста этой работы, мы концентрируемся только на основных чертах и правилах, касающихся видов глагола и их употребления, на уровне, который нужен для понимания главного предмета этой работы. 
%В главе обсуждаются типы видового противопоставления, их семантические подробности и особенности выбора вида

В третьей главе описываются некоторые типы инфинитивных предложений.  Анализ этих предложений ведётся, как указано выше, с точки зрения модальных значений в связи с выбором вида глагола. Показываются примеры значений необходимости и (не)возможности и сравниваем их с указанными нами во второй главе правилами.

