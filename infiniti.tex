\section{Вид глагола в инфинитивном предложении}



--- Infinitiivilause === Инфинитивное предложение - одно из средств выражения модальных значений.  Modaaliset merkitykse обозначает возможное, невозможное, необходимое, неизбежное действие, в инфинитивном предложении действующее лицо побуждается к активному действию (отмечается желательность, необходимость). Деятель (определенное, неопределенное, обобщенное.

Необходимость, невозможность, неизбежность, предстоящее действие, побудительное наклонение.

В грамматике 1980-го года (АГ-80: с. в инфинитивном предложении отмечается всего 11 семантических типов инфинитивного предложения. Нам в этой работе, однако, важно только 


Здесь не пройти, пройти, здесь не переходить, не курить, машину не ставить

Предметом изучения является именно вид глагола в этих предложениях и то, какое значение выбранный вид в том или ином предложении может выразить. Более того, не будем притрагивать средств выражения модальных значений, которых не видно в письменной форме высказывания --- это, прежде всего, интонационные конструкции. Однако рассматриваем различия между СВ и НСВ в побудительных предложениях типа \textit{Встать! Молчать!}.

%АГ-80 \todo{Lue ja referoi modalnost tuosta sivulta 214-215 kahdessa kolmessa kappaleessa}

%Модальность -- это одна из языковых универсалии. Каждое высказывание имеет кое-какие модальные значения. %Модальные значения -- они представляют собой многие варианты. Модальность - это желаемость, необходимость, побуждителтьность, изъявительность, приказание. Модальность можно определить, как 

%%% Построения с глагольными формами и, ei vaikuta relevantilta
%См. также с 231 - интонация как средство выражения субъективно-модальных значений

\todo{jatka tähän osa kuvailua infinitiivilauseissa, s. 373}


\subsection{Значение возможности}


Здесь не пройти 

здесь не курить, машины не ставить



Значение возможности в инфинитивном предложении выражается исключительно совершенным видом. В этом употреблении совершенный вид воспринимается в его результативно-фактическое значении, т.е., СВ высказывает, что результат указанного глаголом действия доступный и его возможно достичь. Предложения с значением невозможности встречается чаще, чем значения возможности (АГ-80: 373). 

В случае значения невозможности при инфинитиве обязательны отрицательные частицы \textit{не}, например \textit{Здесь не пройти. Ему не разобраться.}, или в открытом ряде ни --- ни, например \textit{этот ущерб ни измерить, ни взвесить} (АГ-80: с. 374). При значение возможности обычно употребляется введение частиц \textit{едва, едва ли}, например \textit{Улицы здесь такие узкие, что едва разминуться двум машинам.} \textit{(Хотя улицы совсем узкие, две машины, однако, смогут друг друга проехать)}



выражается совершенным видом

\subsection{Значение необходимости}

Необходимость -- несовершенный вид

мне работать до полуночи
"Я должен работать"

\subsection{Предстоящее действие}

мне скоро уезжать

Предложениями значением возможности соотв

\subsection{Побуждение}

\subsection{Прочие замечания}

Кроме обсуждаемых выше случаев в русском языке встречается независимый инфинитив иногда и в других обстоятельствах. В этих предложениях Совершенным видом можно передать отношения обусловленности и последовательности. 

