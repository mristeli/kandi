\section{Вид глагола в инфинитивном предложении}

В этой главе рассматриваем выражения некоторых модальных значений с точки зрения вида глагола. Попытаемся представить примеры разных типов инфинитивных предложений и описать обстоятельства, в которых встречается той или иной вид глагола. В этих предложениях изучаем то, какое значение выбранный вид в предложении может выразить, и почему именно тот вид является для контекста логичным. 

%Более того, не будем обсуждать средств выражения модальных значений, не видимых в письменной форме высказывания --- это, прежде всего, интонационные конструкции. Однако рассматриваем различия между совершенным и несовершенным видами в побудительных предложениях типа \textit{Встать! Молчать!}

В грамматике 1980-го года (АГ-80 т. 2: с. 373 -- 376) отличаются всего 11 семантических типов инфинитивного предложения. Модальные значения подразделяются на объективную и субъективную предопределенность. Объективная предопределенность --- неизбежность, долженствование, предстояние, т. е., в предложениях со значением объективной предопределенности говорящий выражает не свое желание, мнение и т. п., а внешнюю обязанность, состояние. А субъективную предопределенность, в свою очередь, конкретизируется как побуждение, желание и т. п., т. е., субъективная предопределенность --- выражение воли говорящего.

Независимого инфинитив встречается в русском языке чаще всего в определенных синтаксических структурах, одному из которых в основном подходит только один из двух видов глагола (Рассудова 1968: с. 54). Ниже изучаем типы употребления видов.

\subsection{Значение возможности}

Значение возможности в инфинитивном предложении выражается исключительно совершенным видом. В этом употреблении совершенный вид воспринимается в его результативно-фактическом значении, т.е., совершенный вид высказывает, что результат действия, обозначенного глаголом, доступный и его возможно достичь. Предложения с значением невозможности встречается чаще, чем значения возможности (АГ-80 т. 2: с. 373). 

В случае значения невозможности при инфинитиве обязательны отрицательные частицы \textit{не}, например \textit{Здесь не пройти. Ему не разобраться.}, или в открытом ряде ни --- ни, например \textit{этот ущерб ни измерить, ни взвесить} (АГ-80 т. 2: с. 374). При значение возможности в утверждениях употребляется введение частиц \textit{едва, едва ли}, например \textit{Улицы здесь такие узкие, что едва разминуться двум машинам.}, что можно перефразировать как \textit{Хотя улицы совсем узкие, две машины могут друг друга проехать.} 

Употребление видов глагола в значении (не)возможности точно соответствует употреблению видов глагола со словами \textit{можно, нельзя}. Например, вопросы типа \textit{Как пройти до красной площади?} можно перефразировать как \textit{Как можно пройти до красной площади?}, и все оттенки останутся такими же. Это значение относится к объективной предопределенности.

\subsection{Побуждение и запрещение}

Независимый инфинитив можно употреблять также в побудительном значении. Побудительное значение выражается подходящей интонационной конструкцией. В таких предложениях можно употреблять как совершенный, так и несовершенный вид. Здесь на выбор вида, прежде всего, влияет видовое значение глагола. Употребляя совершенного вида, говорящий хочет достичь результата, побудить другого человека совершить действие, обозначаемое глаголом, целиком. Несовершенный вид употребляется при контексте, в котором внимание направляется на процесс, который другой человек должен либо начать, либо продолжать. К примеру, с инфинитивом совершенного вида \textit{Встать!} говорящий хочет, чтобы лежащие стали стоящими, совершили превращение состояние. Соответственно с глаголо несовершенного вида \textit{Стоять!} говорящий хочет, чтобы люди остались стоящими на месте, продолжали стоять по-прежнему. (АГ-80 т. 2: с. 374.) 

В отрицательных предложениях подобного типа употребляется только несовершенный вид. Это употребление несовершенного вида соответствует употреблению инфинитива несовершенного вида в отрицательных конструкциях типа \textit{не стоит, не следует} или в соответствующей конструкции в повелительном наклонении: \textit{Не рассказывайте об этом! Не опаздывай!}. В этих предложениях инфинитив несовершенного вида обозначает запрещение, и то, что возможен только несовершенный вид, оказывается совсем логичным, поскольку действие, обозначенное глаголом целиком является нежелаемым для говорящего. Значения побуждения и запрещения относятся к субъектной предопределенности. (там же: с. 374)

\subsection{Значение необходимости / предстоящего действия}

У несовершенного вида есть также значение необходимости или предстоящего действия в инфинитивном предложении. Это значение ощущается в предложениях типа \textit{Тебе скоро уезжать (Рассудова 1968: с. 62).} Данный пример можно перефразировать как \textit{ты должен скоро уехать}. Для таких предложений характерен семантический субъект в дательном падеже (в данном примере \textit{мне}). Субъект, однако, может и отсутствовать от предложения: \textit{Завтра вставать рано}, и в таких случаях субъект действия совпадает с говорящим (ТФГ: с. 149). Значение необходимости относится к субъективной предопределенности.

\subsection{Прочие замечания}

Кроме обсуждаемых выше случаев в русском языке встречается независимый инфинитив иногда и в других обстоятельствах. В сложно-подчинённых предложениях, независимый инфинитив совершенного вида может поступать в обобщенно-личном контексте: \textit{Если его попросить, он непременно поможет. Прежде чем лечь, мы долго разговаривали.} В этих предложениях отношения обусловленности и последовательности выражаются совершенным видом. (Рассудова 1978: с. 55.)   

Несовершенный вид в форме инфинитива встречается часто в разговорной речи также в общефактическом значении, более того с вопросительными словами. В этих случаях от контекста уже ясно, что речь идёт о единичном действии, и центр внимания в сообщении тогда уже не на глаголе, а на вопросительном слове. Поэтому возможен и глагол несовершенного вида. Например, \textit{--- мы должны провести подобный эксперимент. --- а когда его проводить?}. Однако, возможно и синонимическое употребление видод: \textit{что тебе взять на второе? --- что тебе брать на второе?} (там же: с. 60 -- 61.)
