\chapter{Модальность и вид глагола в инфинитивном предложении}

Модальность -- языковая 

\subsection{Средства выражения модальных значений}



АГ-80 \todo{Lue ja referoi modalnost tuosta sivulta 214-215 kahdessa kolmessa kappaleessa}

Модальность -- это одна из языковых универсалии. Каждое высказывание имеет кое-какие модальные значения. Модальные значения -- они представляют собой многие варианты. Модальность - это желаемость, необходимость, побуждителтьность, изъявительность, приказание. Модальность можно определить, как 

Модальные значения в русском языке выражаются рядом способов: интонационные конструкции, модальные слова, глагольные формы и т.д. Нише будет обсуждаться две стороны модальности, (не)возможность и необходимость в инфинитивных предложениях.

Средства формировании и выражения субъективно-модальных значений
%%% Построения с глагольными формами и, ei vaikuta relevantilta
%См. также с 231 - интонация как средство выражения субъективно-модальных значений

\todo{jatka tähän osa kuvailua infinitiivilauseissa, s. 373}

\subsection{Значение возможности}

Tee tähän esimerkkejä АГ-80:stä.

\subsection{Значение необходимости}

Tee tähän esimerkkejä АГ-80:stä.
