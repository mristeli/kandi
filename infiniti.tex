\section{Вид глагола в инфинитивном предложении}

В этой главе рассматриваем выражения этих значений с точки зрения вида. Попытаемся представить примеры разных типов инфинитивных предложений и описать обстоятельства, в которых встречается той или иной вид глагола. В этих предложениях изучаем то, какое значение выбранный вид в том или ином предложении может выразить. Более того, не будем обсуждать средств выражения модальных значений, не видимые в письменной форме высказывания --- это, прежде всего, интонационные конструкции. Однако рассматриваем различия между СВ и НСВ в побудительных предложениях типа \textit{Встать! Молчать!}.

--- Infinitiivilause === Инфинитивное предложение - одно из средств выражения модальных значений.  Modaaliset merkitykse обозначает возможное, невозможное, необходимое, неизбежное действие, в инфинитивном предложении действующее лицо побуждается к активному действию (отмечается желательность, необходимость). Деятель (определенное, неопределенное, обобщенное.

Необходимость, невозможность, неизбежность, предстоящее действие, побудительное наклонение. В грамматике 1980-го года (АГ-80: с. в инфинитивном предложении отмечается всего 11 семантических типов инфинитивного предложения. В ней модальные значения подразделяются на объективную и субъективную предопределенность. Объективная предопределенность --- неизбежность, долженствование, предстояние, кратко говоря, в предложениях со значением объективной предопределенности говорящий выражает не свое желание, мнение и т. п., а общий факт. Субъективную предопределенность, в свою очередь конкретизируется как побуждение, желание, разные выражение воли.


\subsection{Значение возможности}

Значение возможности в инфинитивном предложении выражается исключительно совершенным видом. В этом употреблении совершенный вид воспринимается в его результативно-фактическом значении, т.е., СВ высказывает, что результат указанного глаголом действия доступный и его возможно достичь. Предложения с значением невозможности встречается чаще, чем значения возможности (АГ-80: 373). 

В случае значения невозможности при инфинитиве обязательны отрицательные частицы \textit{не}, например \textit{Здесь не пройти. Ему не разобраться.}, или в открытом ряде ни --- ни, например \textit{этот ущерб ни измерить, ни взвесить} (АГ-80: с. 374). При значение возможности обычно употребляется введение частиц \textit{едва, едва ли}, например \textit{Улицы здесь такие узкие, что едва разминуться двум машинам.} \textit{(Хотя улицы совсем узкие, две машины в любом случае смогут друг друга проехать)}

Употребление видов глагола в значении (не)возможности точно соответствует употреблению видов глагола со словами \textit{можно, нельзя}. Данное значение относится к значениям объективной предопределенности. 

здесь не курить, машины не ставить
выражается совершенным видом

\subsection{Значение необходимости}

Необходимость -- несовершенный вид
Отсутствие необходимости

мне работать до полуночи
"Я должен работать"

\subsection{Предстоящее действие}

Отмечается ещё один тип инфинитивного действия в утвер

мне скоро уезжать


Предложениями значением возможности соотв

\subsection{Побуждение}

Встать! говорящий хочет, чтобы лежащие стали стоящими, станут стоять (результативность)

Стоят! говорящий хочет, чтобы люди остались на месте, не перестанут стоять (продолжение процесса)

\subsection{Прочие замечания}

Кроме обсуждаемых выше случаев в русском языке встречается независимый инфинитив иногда и в других обстоятельствах. В этих предложениях Совершенным видом можно передать отношения обусловленности и последовательности. 

