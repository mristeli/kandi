\section{Вид глагола в инфинитивном предложении}

Модальные значения в русском языке выражаются рядом способов: интонационные конструкции, модальные слова, глагольные формы и т. д. Модальные  (АГ-80: с. 214)


Нише обсуждаеться три модальные разновидность --- (не)возможность, необходимость и предстоящее действие (неизбежность) в инфинитивных предложениях.
Тимофеев (1950: с. 267) подразделяет инфинитивные предложения по составу сказуемого также на три: Собственно-инфинитивные предложения без частицы "бы", собственно-инфинитивные предложения с данной частицей, глагольно-инфинитивные предложения с формами глагола "быть". Эти группы отличаются именно модальными значениях. Согласно Тимофееву, собственно-инфинитивные предложения без частицы "бы" в основном выражают долженствование и необходимость, подобные предложения с данной частицей долженствование в более желательном тоне. А третья группа состоит из предложений, в которых в настоящем времени можно ощутить невидимую форму глагола быть, которого, как известно, в современном русском языке не употребляется.

В рамках этой работы рассматриваются только предложения первого типа, предложения с независимым инфинитивом без служебных слов, влияющих на восприятие модальных значений данных предложений. Предметом изучения является именно вид глагола в этих предложениях и то, какое значение выбранный вид в том или ином предложении может выразить. Более того, не будем притрагивать средств выражения модальных значений, которых не видно в письменной форме высказывания --- это, прежде всего, интонационные конструкции, от которых происходить различия между изъявительными, повелительными и вопросительными предложениями, и поэтому ограничиваемся указанными выше тремя значениями. 

%АГ-80 \todo{Lue ja referoi modalnost tuosta sivulta 214-215 kahdessa kolmessa kappaleessa}

%Модальность -- это одна из языковых универсалии. Каждое высказывание имеет кое-какие модальные значения. %Модальные значения -- они представляют собой многие варианты. Модальность - это желаемость, необходимость, побуждителтьность, изъявительность, приказание. Модальность можно определить, как 

%%% Построения с глагольными формами и, ei vaikuta relevantilta
%См. также с 231 - интонация как средство выражения субъективно-модальных значений

\todo{jatka tähän osa kuvailua infinitiivilauseissa, s. 373}


\subsection{Значение необходимости}


\subsection{Значение возможности}

Значение возможности в инфинитивном предложении выражается исключительно совершенным видом. В этом употреблении совершенный вид воспринимается в его результативно-фактическое значении, т.е., СВ высказывает, что результат указанного глаголом действия доступный и его возможно достичь. Предложения с значением невозможности встречается чаще, чем значения возможности (АГ-80: 373). 

В случае значения невозможности при инфинитиве обязательны отрицательные частицы \textit{не}, например \textit{Здесь не пройти. Ему не разобраться.}, или в открытом ряде ни --- ни, например \textit{этот ущерб ни измерить, ни взвесить} (АГ-80: с. 374). При значение возможности обычно употребляется введение частиц \textit{едва, едва ли}, например \textit{Улицы здесь такие узкие, что едва разминуться двум машинам.} \textit{(Хотя улицы совсем узкие, две машины, однако, могут друг друга проехать)}
\subsection{Прочие замечания}

Кроме обсуждаемых выше случаев в русском языке встречается независимый инфинитив иногда и в других обстоятельствах. В этих предложениях Совершенным видом можно передать отношения обусловленности и последовательности. 