\newcommand{\indenttext}{\hspace*{4ex}}
\newcommand{\cutline}{\\\indenttext}
\newcommand{\cutword}{-\cutline}

\section*{Список использоваемой литературы}
\setlength{\parindent}{0cm}
\begin{normalsize}
\subsection*{Исследовательская литература}
АГ-80 = \textit{Русская грамматика I \& II. Синтаксис.} Главный ред. Н.Ю. Шве\cutword дова. М.: Наука, 1980.\\
ЛИТ-10 = \textit{Современный русский литературный язык.} Под ред. В.Г.\cutline Костомарова, В.И. Максимова. М.: Юрайт, 2010.\\
СКОБЛИКОВА 2006: Скобликова Е.С. \textit{Современный русский язык. Син\cutword таксис простого предложения. Учебное пособие} М.: Флинта \& Наука, \indenttext 2006.\\
CОВ-02 = \textit{Современный русский язык: Теория. Анализ языковых единиц.\cutline Ч. 2. Морфология. Синтаксис.} Под ред. Е.И. Дибровой. М.:\cutline Академия, 2002.\\
ТИМОФЕЕВ 1950: Тимофеев, К.А. Об основных типах инфинитивных пред-\indenttext ложений в современном русском литературном языке. \textit{Вопросы синтак\cutword сиса современного русского языка.} Под ред. В.В. Виноградова. М.: Гос.\cutline учебно-педагог. изд-во, 1950, 257--301.\\
ТФГ = \textit{Теория функциональной грамматики т. 2: Темпоральность, \cutline Модальность.} Под ред. А.В. Бондаренко. Л.: Наука, 1990.\\
ШАТУНОВСКИЙ 2009: Шатуновский, И.Б. \textit{Проблемы русского вида.} \cutline М.: Языки славянских культур, 2009.
\end{normalsize}