\newcommand{\indenttext}{\hspace*{4ex}}
\newcommand{\cutline}{\\\indenttext}
\newcommand{\cutword}{-\cutline}

\section*{Список использоваемой литературы}
\setlength{\parindent}{0cm}
\begin{normalsize}
\subsection*{Исследовательская литература}
АГ-80 = \textit{Русская грамматика II. Синтаксис.} Главный ред. Н.Ю.\cutline Шведова. М.: Наука, 1980.\\
ВИНОГРАДОВ 1975 [1960]: Виноградов, В.В. Основные вопросы синтак\cutword сиса предложения. В.В Виноградов. \textit{Избранные труды: Исследования\cutline по русской грамматике.} М.: Наука.\\
ГЛОВИНСКАЯ 1982: Гловинская, М.Я. \textit{Семантические типы видовых\cutline противопоставлений русского глагола.} М.: Наука.\\
ЛИТ-10 = \textit{Современный русский литературный язык.} Под ред. В.Г.\cutline Костомарова, В.И. Максимова. М.: Юрайт, 2010.\\
КАЛИНИН 1994: Калинин, А.Ф. \textit{Односоставные предложения в совре\cutword менном русском языке.} Балашов: Изд-во БГПИ. \\
РАССУДОВА 1968: Рассудова, О.П. \textit{Употребление видов глагола в рус\cutline ском языке.} М.: Изд-во Московского университета.\\
СКОБЛИКОВА 2006: Скобликова, Е.С. \textit{Современный русский язык. Син-\indenttext таксис простого предложения. Учебное пособие.} М.: Флинта \& На\cutword ука.\\
CОВ-02 = \textit{Современный русский язык: Теория. Анализ языковых единиц.\cutline Ч. 2. Морфология. Синтаксис.} Под ред. Е.И. Дибровой. М.:\cutline Академия, 2002.\\
ТИМОФЕЕВ 1950: Тимофеев, К.А. Об основных типах инфинитивных \indenttext предложений в современном русском литературном языке. \textit{Вопросы \cutline синтаксиса современного русского языка.} Под ред. В.В. Виноградо\cutword ва. М.: Гос. учебно-педагог. изд-во, 257--301.\\
ТФГ = \textit{Теория функциональной грамматики т. 2: Темпоральность, \cutline Модальность.} Под ред. А.В. Бондаренко. Л.: Наука, 1990.\\
%ШАТУНОВСКИЙ 2009: Шатуновский, И.Б. \textit{Проблемы русского вида.} \cutline М.: Языки славянских культур.\\
OJANEN 1994: Ojanen, M. \textit{Грамматика: Venäjän kielioppi.} Juva: 1994 
\end{normalsize}