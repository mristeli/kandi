\section*{Список использоваемой литературы}
\setlength{\parindent}{0cm}
\begin{normalsize}
\subsection*{Исследовательская литература}
АГ-80 = \textit{Русская грамматика II. Синтаксис.} Главный ред. Н.Ю. Шведова.\\\hspace*{4ex}М.: Наука, 1980.\\
СОВРУС = \textit{Современный русский литературный язык.} Под ред. В.Г.\\\hspace*{4ex}Костомарова, В.И. Максимова. М.: Юрайт, 2010.\\
ТИМОФЕЕВ 1950: Тимофеев, К.А. Об основных типах инфинитивных пред-\hspace*{4ex}ложений в современном русском литературном языке. \textit{Вопросы синтак-\hspace*{4ex}сиса современного русского языка.} Под ред. В.В. Виноградова. М.: Гос. \hspace*{4ex}учебно-педагог. изд-во, 1950, 257-301.\\
ТФГ = \textit{Теория функциональной грамматики т. 2: Темпоральность, \\\hspace*{4ex}Модальность.} Под ред. А.В. Бондаренко. Л.: Наука, 1990.\\
ШАТУНОВСКИЙ 2009: Шатуновский, И.Б. \textit{Проблемы русского вида.} \\\hspace*{4ex}М.: Языки славянских культур, 2009. 

%Бондарко А. В. Теория функциональной грамматики: Темпоральность, модальность. – Наука, 1990. – Т. 2.


\end{normalsize}